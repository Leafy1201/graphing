\documentclass[../../../../main.tex]{subfiles}
\begin{document}

\section{Functions}
At the moment we only need a class for explicit functions, however to illustrate the reason behind creating a separate class I will also create a quick Normal Distribution function that extends our class for explicit functions. This will be more useful later on when we implement layers.

\subsection{Explicit Function}
This class will essentially be a wrapper, a layer to manage the expression objects underneath. 
\begin{minted}[
frame=lines,
framesep=2mm,
linenos,
breaklines
]{java}
package structures;

import exceptions.StackOverflowException;
import exceptions.StackUnderflowException;
import exceptions.UnequalBracketsException;

public class ExplicitFunction {

	// attributes
	private Expression f; // the actual mathematical function

	// methods
	// empty constructor to allow for inheritance
	public ExplicitFunction() {
	}

	// constructor
	public ExplicitFunction(String expression, char parameter) throws StackOverflowException, StackUnderflowException, UnequalBracketsException {
		this.setF(new Expression(expression, parameter));
	}

	// evaluate method calls the expression evaluate method
	public double evaluate(double x) {
		// try to evaluate the expression
		//if it fails simply return NaN and the associated error
		try {
			return f.evaluate(x);
		} catch (StackUnderflowException e) {
			System.out.println("underflow");
		} catch (StackOverflowException e) {
			System.out.println("overflow");
		} catch (NumberFormatException e) {
			System.out.println("multiple parameters");
		}
		return Double.NaN;
	}




	// set the expression attribute - used for child classes
	protected void setF(Expression f) {
		this.f = f;
	}

	@Override
	// output the representation of the expression object
	public String toString() {
		return this.f.toString();
	}

}
\end{minted}
To test this class I used the following code with comments for what I expected:
\begin{minted}[breaklines]{java}
ExplicitFunction f = new ExplicitFunction("x^2",'x');
	// valid data
	System.out.println(f.evaluate(2.5));	// expected result: 6.25
\end{minted}
Here are the results:
\begin{minted}[breaklines]{console}
6.25
\end{minted}
Which is exactly what I expected.
\subsection{Normal Distribution Function}
This class will inherit the Explicit Function class and will basically have its own expression that we will substitute the parameters, mean, $\mu$, and variance, $\sigma^2$, into.
\begin{minted}[
frame=lines,
framesep=2mm,
linenos,
breaklines
]{java}
package structures;

import exceptions.StackOverflowException;
import exceptions.StackUnderflowException;
import exceptions.UnequalBracketsException;

public class NormalDistribution extends ExplicitFunction {

	// attributes
	private double μ; // mean
	private double σ2; // variation

	// methods
	// constructor
	public NormalDistribution(double μ, double σ2)
			throws StackOverflowException, StackUnderflowException, UnequalBracketsException {
		// super constructor must be called first in Java
		super();

		// set the attributes
		this.μ = μ;
		this.σ2 = σ2;

		// create the expression for the normal distribution function
		String expression = "(1/(2*pi*σ2)^(1/2))*e^(-(x-μ)^2/(2*σ2^(1/2)))";
		// substitute the mean and variance values
		expression = expression.replace("σ2", Double.toString(σ2));
		expression = expression.replace("μ", Double.toString(μ));

		// set the new expression
		this.setF(new Expression(expression, 'x'));
	}

	// return the mean
	public double getΜ() {
		return μ;
	}

	// return the variance
	public double getΣ2() {
		return σ2;
	}

	// return the standard deviation
	public double getΣ() {
		return Math.sqrt(σ2);
	}

}
\end{minted}
To test this class I used the following code with comments for what I expected:
\begin{minted}[breaklines]{java}
ExplicitFunction f = new NormalDistribution(0,1);
	// valid data
	System.out.println(f.evaluate(1));	// expected value: 0.24197 to 5 s.f.
\end{minted}
Here are the results:
\begin{minted}[breaklines]{console}
0.24197072451914337
\end{minted}
Which is exactly what I expected.
\newpage

\end{document}