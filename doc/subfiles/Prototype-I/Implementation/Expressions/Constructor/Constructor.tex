\documentclass[../../../../main.tex]{subfiles}
\begin{document}
\subsection{Constructor}
Now that the \texttt{createTree()} function is implemented, the constructor can finally be implemented. Along with the constructor, I will override the \texttt{(toString()} method so that I can view the contents of an object of this type by calling a single method. It keeps the attributes private, and removes the need to have getters and setters for those attributes just to view them. However the most important role it fulfills is avoiding the repetition of code whenever we need to view the inside of this object, which is a good programming practice that also makes our code more organized and clean.

\begin{minted}[
frame=lines,
framesep=2mm,
linenos,
breaklines
]{java}
// constructor
public Expression(String expression, char parameter) throws StackOverflowException, StackUnderflowException, UnequalBracketsException {
	this.parameter = parameter; // sets the parameter value
	this.expression = standardize(expression); // removes whitespace + standardizes
	BinaryTree tree = createTree(this.expression); // creates the tree
	this.postFixStack = tree.traverse(); // creates the post-fix stack
}

@Override
// output the stack and the expression this object represents
public String toString() {
	return this.expression + " -> " + this.postFixStack.toString();
}
\end{minted}
To test these methods I used the following code with comments for what I expected:
\begin{minted}[breaklines]{java}
Expression ex = new Expression("2x", 'x');
System.out.println(ex);	// expected result: 2*x -> [*], x, 2
\end{minted}
Here are the results:
\begin{minted}[breaklines]{console}
2*x -> [*], x, 2
\end{minted}
Which is exactly what I expected.
\newpage
\end{document}