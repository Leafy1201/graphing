\documentclass[../../../../main.tex]{subfiles}
\begin{document}
\subsection{Evaluate}
This function is linked to the class so it will not be \texttt{static}, however it will only be used outside the class since other modules in our program will need to know the value of the expression at a specific value, so it will be \texttt{public}. There are some language specific things that I added to this method. If the current value I have popped of the stack is not an operator, the algorithm casts it to an integer. However this can cause issues, since if we have an expression $xy$, this contains two parameters but we take only one parameter. So say our parameter is $x$, then the $y$ will be treated as a number since it is not an operator, and it will not have been substituted for a value. So when we cast this value we will get an error. To remedy this, my method will be flagged as throwing the related exception \texttt{NumberFormatException}. The usual problems in arithmetic such as dividing by 0, are handled by \texttt{Java double} primitive and the IEEE standard\cite{doubleJava, doubleIEEE}.
\begin{minted}[
frame=lines,
framesep=2mm,
linenos,
breaklines
]{java}
// evaluate the expression for a value of the parameter
public double evaluate(double x) throws StackUnderflowException, StackOverflowException, NumberFormatException {
	// create a stack to store the numerical values
	// and the substitute stack
	Stack<String> subStack = substitute(x);
	Stack<Double> numStack = new Stack<Double>(this.postFixStack.getHeight());
	for (int i = subStack.getHeight(); i > 0; i--) {
		String pop = subStack.pop(); // pop a value of the substitute stack and store it
		double a, b; // initialize two double variables for potential calculation
		switch (pop) { // check if the popped value is a number or operator
		// if it is an operator
		// do the operation on the two numbers at the top of the number stack
		// and replace them both with the result
		case "+":
			b = numStack.pop();
			a = numStack.pop();
			numStack.push(a + b);
			break;
		case "-":
			b = numStack.pop();
			a = numStack.pop();
			numStack.push(a - b);
			break;
		case "*":
			b = numStack.pop();
			a = numStack.pop();
			numStack.push(a * b);
			break;
		case "/":
			b = numStack.pop();
			a = numStack.pop();
			numStack.push(a / b);
			break;
		case "^":
			b = numStack.pop();
			a = numStack.pop();
			numStack.push(Math.pow(a, b));
			break;
		// if it is a number push it onto the number stack
		default:
			// try to cast the popped value, it is of type String, to double
			// this may fail so the function throws NumberFormatException
			// this will fail if there are multiple parameters e.g. "xy"
			// since it will treat the y as a number which it is not
			numStack.push(Double.valueOf(pop));
			break;
		}
	}
	// pop and return the remaining value in the number stack, this is the evaluated value
	return numStack.pop();
}
\end{minted}
To test these methods I used the following code with comments for what I expected:
\begin{minted}[breaklines]{java}
	Expression ex1 = new Expression("2x + 4", 'x');
	System.out.println(ex1.evaluate(3));	//expected value: 10
	
	Expression ex2 = new Expression("e^a", 'a');
	System.out.println(ex2.evaluate(0));	//expected value: 1
	
	Expression ex3 = new Expression("1/b", 'b');
	System.out.println(ex3.evaluate(2));	//expected value: 0.5
	System.out.println(ex3.evaluate(0));	//will output infinity, since we are dividing by 0
	
	Expression ex4 = new Expression("c/c", 'c');
	System.out.println(ex4.evaluate(1));	//expected value: 1
	System.out.println(ex4.evaluate(0));	//will output NaN - not a number, since we are dividing 0 by 0
	
	Expression ex5 = new Expression("xy", 'x');
	System.out.println(ex5.evaluate(2));	//will throw an exception, since there are multiple parameters
\end{minted}
Here are the results:
\begin{minted}[breaklines]{console}
10.0
1.0
0.5
Infinity
1.0
NaN
Exception in thread "main" java.lang.NumberFormatException: For input string: "y"
	at sun.misc.FloatingDecimal.readJavaFormatString(FloatingDecimal.java:2043)
	at sun.misc.FloatingDecimal.parseDouble(FloatingDecimal.java:110)
	at java.lang.Double.parseDouble(Double.java:538)
	at java.lang.Double.valueOf(Double.java:502)
	at structures.Expression.evaluate(Expression.java:76)
\end{minted}
Which is exactly what I expected.
\newpage
\end{document}