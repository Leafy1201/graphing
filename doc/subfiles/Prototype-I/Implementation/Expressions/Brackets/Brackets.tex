\documentclass[../../../../main.tex]{subfiles}
\begin{document}
\subsection{Check Brackets}
Like the \texttt{standardize()} function, this method is related to the class but not to a specific object hence it will be \texttt{static}. Here is the code:
\begin{minted}[
frame=lines,
framesep=2mm,
linenos,
breaklines
]{java}
// check for and remove enclosing matching brackets
private static String checkBracket(String input) throws UnequalBracketsException, StringIndexOutOfBoundsException {
	boolean done = false;
	// check if there are an equal number of opening and closing brackets
	if((input.length() - input.replace("(", "").length())
			!=(input.length() - input.replace(")", "").length())) {
		throw new UnequalBracketsException(input);
	}
	while(!done) {	// repeat until there are no enclosing brackets
		done = true;// assume no enclosing matching brackets until otherwise found
		// check if there are enclosing brackets (not necessarily matching)
		if((input.charAt(0) == '(') && (input.charAt(input.length()-1) == ')')) {
			int countMatching = 1;	// initialize the bracket count as 1 to count for the opening bracket 
			// loop from the second character to one before the end
			for(int i=1; i<input.length() - 1; i++) {
				if (countMatching == 0) {	// if we find the matching bracket before the end terminate and return the input
					return input;
				} else if(input.charAt(i) == '(') {
					// increment if we find an opening bracket
					countMatching++;
				} else if(input.charAt(i) == ')') {
					// decrement if we find an closing bracket
					countMatching--;
				}			
			}
			// if the last character is the matching bracket keep the outer while loop going and and remove the enclosing bracket
			if(countMatching == 1) {	
				input = input.substring(1, input.length() - 1);
				done = false;
			}
		}
	}
	return input;
}
\end{minted}
While this algorithm is explained in the design section, the confusing part of this is the checking if there are an equal number of opening and closing brackets. There is no built-in method for this, but we can use built-in methods to achieve the same result. To do this\cite{countInstanceStringJava} we replace all of the character that we want with empty characters and then take this away from the length of the original String.\footnote{Let $l$ be the original length of the String, $x$ the number of instances of the character and $c$ the length of the reduced String:\[l = x + c \implies x = l - c\]}
\newpage\noindent
To test this method I used the following code with comments for what I expected:
\begin{minted}[breaklines]{java}
System.out.println(checkBracket("2x"));	// 2x - nothing changes
System.out.println(checkBracket("((x))"));	// x - 2 sets of brackets removed
System.out.println(checkBracket("(x-1)(x+1)"));	// (x-1)(x+1) - nothing changes
System.out.println(checkBracket("((x-1)(x+1))")); // (x-1)(x+1) - outer set of bracket removed
System.out.println(checkBracket("((x)"));	// throws an exception
\end{minted}
Here are the results:
\begin{minted}[breaklines]{console}
2x
x
(x-1)(x+1)
(x-1)(x+1)
Exception in thread "main" exceptions.UnequalBracketsException: There are an unequal number of opening and closing brackets in the expression: (()
	at structures.Expression.checkBracket(Expression.java:133)
	at structures.Expression.main(Expression.java:274)
\end{minted}
Which is exactly what I expected.\footnote{Like with the \texttt{Stack} class I  created a custom exception for this.}
\newpage
\end{document}