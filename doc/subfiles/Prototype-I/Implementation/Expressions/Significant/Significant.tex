\documentclass[../../../../main.tex]{subfiles}
\begin{document}
\subsection{Least Significant Operator}
Again this method is not linked an object, hence it will be static. The code is shown below:
\begin{minted}[
frame=lines,
framesep=2mm,
linenos,
breaklines
]{java}
//find the least significant operator in an expression
private static int leastSigOperatorPos(String input) {
	int parenthesis = 0;
	int leastSigOperatorPos = -1;
	int leastSigOpcode = 1000;	//make the opcode super high so every operator is more significant than it
	//used a string to store the operators as it is essentially a char array but with added utility such as finding elements
	final String operators = "+-*/^";
	int currentOpcode;
	//loop through the entire input
	for(int i=0; i<input.length(); i++) {
		char currentChar = input.charAt(i);
		currentOpcode = operators.indexOf(currentChar);
		//check if it is an operator
		if(currentOpcode>=0) {
			//check if it the same or more significant and if it is not enclosed in parenthesis
			if((currentOpcode <= leastSigOpcode) && (parenthesis == 0)) {
				//update the significance and the position of the operator
				leastSigOperatorPos = i;
				leastSigOpcode = currentOpcode;
			}
		} else if(currentChar == '(') {
			//increment to signify we have entered an enclosing bracket
			parenthesis++;
		} else if(currentChar == ')') {
			//decrement to signify we have exited an enclosing bracket
			parenthesis--;
		}
	}
	return leastSigOperatorPos;
}
\end{minted}
To test this method I used the following code with comments for what I expected:
\begin{minted}[breaklines]{java}
System.out.println(leastSigOperatorPos("3+x^2"));	//should return 1 for the +
System.out.println(leastSigOperatorPos("(x+3)"));	//should return -1 since it is enclosed in brackets
System.out.println(leastSigOperatorPos("x*(x+3)^3"));	//should return 1 for the *
\end{minted}
Here are the results:
\begin{minted}[breaklines]{console}
1
-1
1
\end{minted}
Which is exactly what I expected.
\newpage
\end{document}