\documentclass[../../../../main.tex]{subfiles}
\begin{document}

\section{Layers}
As described in the design section, I will create classes that inherit from one a top class that I will call \texttt{Layer}. This class will be \texttt{abstract} since we will never instantiate this class and only inherit from it. The \texttt{CartesianLayer} will be a child class to the abstract \texttt{Layer} class. Again this will be \texttt{abstract} since we will only inherit from it. While it seems redundant for the super class \texttt{Layer} to exist, it fulfils the purpose of being able to create an array of type \texttt{Layer} which can then be filled with objects that are of types which are child classes to \texttt{Layer}. Our first instantiable class will be a child class to \texttt{CartesianLayer} and will be called \texttt{ExplicitXFunctionCartesianLayer}. The \texttt{X} signifies that the parameter will be $x$. Likewise we will have a class called \texttt{ExplicitYFunctionCartesianLayer}, where this will be a function in terms of $y$. These two function classes are very similar, they both use algorithm \ref{alg:drawFunc} on page \pageref{alg:drawFunc} but one uses the primary variable $x$ and the other uses the primary variable $y$. Our \texttt{AxesCartesianLayer} will also inherit from the \texttt{CartesianLayer} parent class. This makes actually drawing the axes easier, since we can simply draw 2 lines using the methods in the parent class, \texttt{CartesianLayer}.
\\A diagram of this inheritance is shown below:

\begin{figure}[H]
\begin{center}
\begin{forest}
  for tree={
    align=center,
    parent anchor=south,
    child anchor=north,
    font=\sffamily,
    edge={thick, -{Stealth[]}},
    l sep+=10pt,
    edge path={
      \noexpand\path [draw, \forestoption{edge}] (!u.parent anchor) -- +(0,-10pt) -| (.child anchor)\forestoption{edge label};
    },
    if level=0{
      inner xsep=0pt,
      tikz={\draw [thick] (.south east) -- (.south west);}
    }{calign=child,calign child=2}
  }
  [Layer
    [CartesianLayer
      [ExplicitXFunctionCartesianLayer]
      [ExplicitYFunctionCartesianLayer]
      [AxesCartesianLayer]
      ]
    [InputLayer] 
    ]
  ]
\end{forest}
\end{center}
\caption{\texttt{Layer} Inheritance Diagram}
\end{figure}

The problem with this part of the program is that everything is interdependent on everything else. Specifically here the \texttt{PlotPane} class is used within the \texttt{Layer} classes and vice versa. Since I will implement the \texttt{Layer} classes first I will create a bare bones  \texttt{PlotPane} class that just contains its attributes for use within the \texttt{Layer} classes and implement the \texttt{PlotPane} class later. This is simply to avoid compiler errors. This also means that we cannot properly test this portion until we complete both parts. As such I will combine the mini tests that I would have carried out, as a part of the overall final testing for this prototype.
\newpage
\subsection{Layer}
This is an abstract super class, as described in the design. Since we cannot instantiate this class, we cannot directly test this class, but by testing its child classes we can test it indirectly. I added an the attribute \texttt{Color} and setter for it, which controls what the color of anything drawn onto the canvas is.
\begin{minted}[
frame=lines,
framesep=2mm,
linenos,
breaklines,
fontsize = \fontsize{9.5}{9.5}
]{java}
package layer;

import application.PlotPane;
import javafx.beans.property.DoubleProperty;
import javafx.beans.property.IntegerProperty;
import javafx.beans.property.SimpleDoubleProperty;
import javafx.beans.property.SimpleIntegerProperty;
import javafx.scene.canvas.Canvas;
import javafx.scene.canvas.GraphicsContext;
import javafx.scene.paint.Color;

public abstract class Layer {

	// attributes
	// drawing tools
	protected Canvas canvas;
	protected GraphicsContext gc;
	protected Color color = Color.BLACK;
	// drawing parameters
	protected IntegerProperty steps = new SimpleIntegerProperty(0);
	protected DoubleProperty minX = new SimpleDoubleProperty(0);
	protected DoubleProperty maxX = new SimpleDoubleProperty(0);
	protected DoubleProperty minY = new SimpleDoubleProperty(0);
	protected DoubleProperty maxY = new SimpleDoubleProperty(0);
	protected DoubleProperty pixelWorthX = new SimpleDoubleProperty(0);
	protected DoubleProperty pixelWorthY = new SimpleDoubleProperty(0);

	// methods
	// constructor
	protected Layer() {
		// instantiate the canvas and create the OpenGL context
		this.canvas = new Canvas();
		this.gc = canvas.getGraphicsContext2D();
	}

	// will draw onto the canvas, will be overridden
	public abstract void draw();

	// will connect this layer to a PlotPane instance, will be overridden
	public abstract void bindProperties(PlotPane plotPane);

	// clear the canvas to be redrawn
	protected void clearCanvas() {
		gc.clearRect(0, 0, canvas.getWidth(), canvas.getHeight());
	}

	// getter for the canvas
	public Canvas getCanvas() {
		return canvas;
	}

	// change the color of the function drawn
	public void setColor(Color color) {
		this.color = color;
	}

}
\end{minted}
\newpage
\subsection{Input Layer}
This class, unlike the previous, is not an abstract class. It was quite simple to implement with the algorithms \ref{alg:pixelWorth}, \ref{alg:funcLayerConstructor}, \ref{alg:inpLayerConstructor} and \ref{alg:layerConstructor} on pages \pageref{alg:pixelWorth} and \pageref{alg:funcLayerConstructor}. In the \texttt{bindProperties()} method I initialized the values for the minimum and maximum values of $x$ and $y$ to go from $-10$ to $10$.
\begin{minted}[
frame=lines,
framesep=2mm,
linenos,
breaklines,
fontsize = \fontsize{9}{9}
]{java}
package layer;

import java.text.DecimalFormat;

import application.PlotPane;
import javafx.beans.property.BooleanProperty;
import javafx.beans.property.DoubleProperty;
import javafx.beans.property.IntegerProperty;
import javafx.beans.property.SimpleBooleanProperty;
import javafx.beans.property.SimpleDoubleProperty;
import javafx.beans.property.SimpleIntegerProperty;

public class InputLayer extends Layer {
	
	// attributes
	private BooleanProperty changeViewport = new SimpleBooleanProperty(true);
	
	// methods
	// constructor
	public InputLayer() {
		// call the super constructor
		super();
		//update pixel worth when the canvas changes size
		this.canvas.heightProperty().addListener(event -> updatePixelWorth());
		this.canvas.widthProperty().addListener(event -> updatePixelWorth());
	}


	@Override
	// this canvas will not draw anything so this method is empty
	// it must still be implemented since the method was abstract in the parent class
	public void draw() {}
	
	// update the pixelworth
	private void updatePixelWorth() {
		this.pixelWorthX.set((this.maxX.doubleValue() - this.minX.doubleValue()) / this.canvas.getWidth());
		this.pixelWorthY.set((this.maxY.doubleValue() - this.minY.doubleValue()) / this.canvas.getHeight());
	}

	@Override
	// initialize the properties required so they can be bound later
	public void bindProperties(PlotPane plotPane) {
		// initialize the properties
		this.steps = new SimpleIntegerProperty(400);
		this.minX = new SimpleDoubleProperty(-10);
		this.maxX = new SimpleDoubleProperty(10);
		this.minY = new SimpleDoubleProperty(-10);
		this.maxY = new SimpleDoubleProperty(10);
		this.pixelWorthX = new SimpleDoubleProperty(1);
		this.pixelWorthY = new SimpleDoubleProperty(1);
		// make the canvas resize as its parent resizes by binding the associated properties
		this.canvas.heightProperty().bind(plotPane.heightProperty());
		this.canvas.widthProperty().bind(plotPane.widthProperty());
	}
	
	//return minX
	public DoubleProperty getMinX() {
		return minX;
	}
	
	//return minX
	public DoubleProperty getMaxX() {
		return maxX;
	}
	
	// return minX
	public DoubleProperty getMinY() {
		return minY;
	}
	
	// return minX
	public DoubleProperty getMaxY() {
		return maxY;
	}
	
	// return the worth of each pixel in the x-direction
	public DoubleProperty getPixelWorthX() {
		return pixelWorthX;
	}
	
	// return the worth of each pixel in the y-direction
	public DoubleProperty getPixelWorthY() {
		return pixelWorthY;
	}

	// return the number of draw steps
	public IntegerProperty getSteps() {
		return steps;
	}
	
	// return the property that notifies when it is time to redraw layers
	public BooleanProperty getChangeViewport() {
		return changeViewport;
	}

}
\end{minted}
\newpage
\subsection{Cartesian Layer}
This class is also abstract and hence cannot be instantiated. It will contain methods to convert Cartesian coordinates to the Canvas coordinate system and bind its properties to a \texttt{PlotPane}. The algorithms for these are on page \pageref{alg:coordConv}. Since all the classes that inherit this class will use the same properties I will finally override the \texttt{bindProperties{}} method and bind the necessary properties to reduce the amount of redundant code and ensure consistency in how the properties are bound.
\begin{minted}[
frame=lines,
framesep=2mm,
linenos,
breaklines,
fontsize = \fontsize{9}{9}
]{java}
package layer;

import application.PlotPane;
import javafx.beans.property.SimpleDoubleProperty;
import javafx.beans.property.SimpleIntegerProperty;

public abstract class CartesianLayer extends Layer {

	@Override
	// must be overridden since the parent class made it abstract
	// but keep it abstract so child class must override
	public abstract void draw();

	@Override
	// bind the needed properties from the input layer to itself
	public void bindProperties(PlotPane plotPane) {
		// initialize the properties with the same values as their partners
		this.steps = new SimpleIntegerProperty(plotPane.getInputLayer().getSteps().intValue());
		this.minX = new SimpleDoubleProperty(plotPane.getInputLayer().getMinX().doubleValue());
		this.maxX = new SimpleDoubleProperty(plotPane.getInputLayer().getMaxX().doubleValue());
		this.minY = new SimpleDoubleProperty(plotPane.getInputLayer().getMinY().doubleValue());
		this.maxY = new SimpleDoubleProperty(plotPane.getInputLayer().getMaxY().doubleValue());
		this.pixelWorthX = new SimpleDoubleProperty(plotPane.getInputLayer().getPixelWorthX().doubleValue());
		this.pixelWorthY = new SimpleDoubleProperty(plotPane.getInputLayer().getPixelWorthY().doubleValue());
		// bind the two properties together one-directionally
		// so these attributes change when plot pane changes, not vice versa
		this.steps.bind(plotPane.getInputLayer().getSteps());
		this.minX.bind(plotPane.getInputLayer().getMinX());
		this.maxX.bind(plotPane.getInputLayer().getMaxX());
		this.minY.bind(plotPane.getInputLayer().getMinY());
		this.maxY.bind(plotPane.getInputLayer().getMaxY());
		this.pixelWorthX.bind(plotPane.getInputLayer().getPixelWorthX());
		this.pixelWorthY.bind(plotPane.getInputLayer().getPixelWorthY());
		// make the canvas resize as its parent resizes by binding the associated properties
		canvas.heightProperty().bind(plotPane.heightProperty());
		canvas.widthProperty().bind(plotPane.widthProperty());
	}

	// convert x-Cartesian coordinate to x-Canvas coordinate
	protected double convertX(double x) {
		x = x - this.minX.doubleValue();
		x = x / this.pixelWorthX.doubleValue();
		return x;
	}

	// convert y-Cartesian coordinate to y-Canvas coordinate
	protected double convertY(double y) {
		y = this.maxY.doubleValue() - y;
		y = y / this.pixelWorthY.doubleValue();
		return y;
	}

}
\end{minted}
\newpage
\subsection{Axes Cartesian Layer}
Drawing the axes is quite simple so we will begin our implementation of non-abstract drawing classes with this one. This class will draw whenever the $x$ or $y$ coordinate is 0. This results in two lines. The $x$ axis will be drawn from the leftmost point where $y$ is 0 to the rightmost point where $y$ is (rightmost and leftmost on the canvas). This means that we draw a line from the point $(x_{min,0})$ to the point $(x_{max,0})$. These $x_min$ and $x_max$ values are values that we have stored as properties within our class. Similarly we can draw the $y$ axis by drawing a line from the point $(0,y_{min})$ to the point $(0,y_{max})$. These Cartesian coordinates must be first converted to the canvas coordinates, and this will be done in the implementation.
\begin{minted}[
frame=lines,
framesep=2mm,
linenos,
breaklines
]{java}
package layer;

public class AxesCartesianLayer extends CartesianLayer {

	// constructor
	public AxesCartesianLayer() {
		// call the super to setup the canvas
		super();
	}

	@Override
	// draw the pair of axes
	public void draw() {
		// clear the canvas before drawing
		this.clearCanvas();
		// make the line width thinner then usual to remove emphasis from axes
		gc.setLineWidth(1.5);
		gc.setStroke(color);
		// draw the y-axis
		gc.strokeLine(this.convertX(0), this.convertY(this.minY.doubleValue()), this.convertX(0),
				this.convertY(this.maxY.doubleValue()));
		// draw the x-axis
		gc.strokeLine(this.convertX(this.minX.doubleValue()), this.convertY(0),
				this.convertX(this.maxX.doubleValue()), this.convertY(0));
	}

}
\end{minted}
\newpage
\subsection{Explicit $x$-Function Cartesian Layer}
While designing the algorithm to draw functions I talked about how the function must be continuous otherwise we would come across arithmetic errors such dividing by 0. This is no longer an issue since the \texttt{double} will handle all these errors for us. This was mentioned during the implementation of the evaluation of a function. Apart from that resolved issue there were no other issues during the implementation of this class.

\begin{minted}[
frame=lines,
framesep=2mm,
linenos,
breaklines
]{java}
package layer;

import exceptions.StackOverflowException;
import exceptions.StackUnderflowException;
import exceptions.UnequalBracketsException;
import structures.ExplicitFunction;

public class ExplicitXFunctionCartesianLayer extends CartesianLayer {

	// attributes
	private ExplicitFunction f;

	// methods
	// constructor - create a layer with an expression given
	public ExplicitXFunctionCartesianLayer(String expression) {
		// super constructor must be invoked first
		super();
		// try to create the function with the given expression
		try {
			this.f = new ExplicitFunction(expression, 'x');
		} catch (StackOverflowException | StackUnderflowException | UnequalBracketsException e) {
			// output the error in parsing the function
			e.printStackTrace();
		}
	}

	// constructor - create a layer with a set function
	public ExplicitXFunctionCartesianLayer(ExplicitFunction f) {
		// super constructor must be invoked first
		super();
		// set the function
		this.f = f;
	}

	// draw the function
	public void draw() {
		// setup the canvas by clearing it and set up the line to be drawn
		this.clearCanvas();
		gc.setLineWidth(2);
		gc.setStroke(color);
		double x1, x2, y1, y2;
		// initialize the first value of x
		x1 = this.minX.doubleValue();
		// evaluate the first value of y
		y1 = f.evaluate(x1);
		// calculate the dx value i.e. the step value
		double step = (this.maxX.doubleValue() - this.minX.doubleValue()) / this.steps.doubleValue();
		// loop until we draw everything in the range, minX to minX, required
		for (x2 = this.minX.doubleValue() + step; x2 < this.maxX.doubleValue(); x2 = x2 + step) {
			// evaluate the next value of y
			y2 = f.evaluate(x2);
			// draw the line
			gc.strokeLine(this.convertX(x1), this.convertY(y1), this.convertX(x2), this.convertY(y2));
			// update the old values
			x1 = x2;
			y1 = y2;
		}
	}

	@Override
	// output the function that this layer is drawing
	public String toString() {
		return this.f.toString();
	}

}
\end{minted}
\subsection{Explicit $y$-Function Cartesian Layer}
This is essentially a copy of the previous layer but we are looping over the $y$ variable instead of the $x$ variable. As such I simply replaced every instance of $X$ with a $Y$ and vice versa.

\begin{minted}[
frame=lines,
framesep=2mm,
linenos,
breaklines
]{java}
package layer;

import exceptions.StackOverflowException;
import exceptions.StackUnderflowException;
import exceptions.UnequalBracketsException;
import structures.ExplicitFunction;

public class ExplicitYFunctionCartesianLayer extends CartesianLayer {

	// attributes
	private ExplicitFunction f;

	// methods
	// constructor - create a layer with an expression given
	public ExplicitYFunctionCartesianLayer(String expression) {
		// super constructor must be invoked first
		super();
		// try to create the function with the given expression
		try {
			this.f = new ExplicitFunction(expression, 'y');
		} catch (StackOverflowException | StackUnderflowException | UnequalBracketsException e) {
			// output the error in parsing the function
			e.printStackTrace();
		}
	}

	// constructor - create a layer with a set function
	public ExplicitYFunctionCartesianLayer(ExplicitFunction f) {
		// super constructor must be invoked first
		super();
		// set the function
		this.f = f;
	}

	// draw the function
	public void draw() {
		// setup the canvas by clearing it and set up the line to be drawn
		this.clearCanvas();
		gc.setLineWidth(2);
		gc.setStroke(color);
		double x1, x2, y1, y2;
		// initialize the first value of y
		y1 = this.minY.doubleValue();
		// evaluate the first value of x
		x1 = f.evaluate(y1);
		// calculate the dy value i.e. the step value
		double step = (this.maxY.doubleValue() - this.minY.doubleValue()) / this.steps.doubleValue();
		// loop until we draw everything in the range, minY to maxY, required
		for (y2 = this.minY.doubleValue() + step; y2 < this.maxY.doubleValue(); y2 = y2 + step) {
			// evaluate the next value of y
			x2 = f.evaluate(y2);
			// draw the line
			gc.strokeLine(this.convertX(x1), this.convertY(y1), this.convertX(x2), this.convertY(y2));
			x1 = x2;
			y1 = y2;
		}
	}

	@Override
	// output the function that this layer is drawing
	public String toString() {
		return this.f.toString();
	}

}
\end{minted}
\newpage
\end{document}