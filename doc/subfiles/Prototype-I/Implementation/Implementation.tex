\documentclass[../../../main.tex]{subfiles}
\begin{document}

\chapter{Implementation}
When beginning to implement my solution, I made a few decisions. I realized a couple of important ideas which I did not address in my Design:
\begin{enumerate}
\item My function class designed was more of a class for an expression instead of a function. This is significant because different functions will use expressions in different ways. For example, an explicit function will use one expression to define the relationship, whereas an implicit function will use two expressions to define this relationship. Therefore to make our solution expandable in the future it would make more sense to have an expression class with multiple other classes for the different types of functions. These function classes can then inherit each other, for example if we make classes for different probability distribution, they are simply explicit functions with a specific standard structure, which makes them perfect child classes.
\item The layer system only relates to the Cartesian Coordinate System. This is a problem if I expand my program later to contain the Polar Coordinate System. So what I will do to fix this problem, is to take the \texttt{Layer} class that I created, rename it to \texttt{CartesianLayer} and make this class inherit a new class called \texttt{Layer} which will be all the attributes some key methods that all layers will have. I will also make another class called \texttt{ExplicitFunctionLayer} which will inherit from \texttt{CartesianLayer}. This allows me to split up each one of my layers into the types of function that will be used.
\item \texttt{Java} allows projects to be split up into packages. To organize my implementation I will split it up into the following packages:
	\begin{enumerate}
	\item \texttt{application} - Contains the UI classes, controllers, \texttt{PlotPane}, etc.
	\item \texttt{layers} - Contains the \texttt{Layer} classes.
	\item \texttt{structures} - Contains the data structure classes, stacks, trees, expressions, functions, etc.
	\item \texttt{exceptions} - Contains custom exceptions.
	\end{enumerate}
\end{enumerate}

\newpage
\end{document}