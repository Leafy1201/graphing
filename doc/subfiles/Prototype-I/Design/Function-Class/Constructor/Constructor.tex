\documentclass[../../../../../main.tex]{subfiles}
\begin{document}

\subsection{Constructor and the Class as a Whole}
The constructor for the \texttt{Function} class will be quite simple. We take an expression as an input and:
\begin{enumerate}
	\item Remove Whitespace from the expression
	\item Standardize the expression (using the RegEx rules)
	\item Create the Binary Tree
	\item Create the Post-Fix Stack from the Binary Tree
\end{enumerate}
\begin{algorithm}
\DontPrintSemicolon
\caption{Function Class Constructor}
\Fn{Function(\KwString expression)}{
	expression = expression.strip()\;
	expression = standardize(expression)\;
	this.expression = expression
	this.binaryTree = createTree(this.expression)\;
	this.postFixStack = this.binaryTree.traverse()\;
}
\end{algorithm}
The class as a whole will look something like the diagram below.
\begin{figure}[H]
	\centering
	\includegraphics[width=0.45\textwidth]{diagrams/function.mps}
	\caption{\texttt{Function} Class}
\end{figure}
\newpage
\end{document}