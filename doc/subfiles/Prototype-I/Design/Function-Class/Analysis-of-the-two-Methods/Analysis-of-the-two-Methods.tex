\documentclass[../../../../../main.tex]{subfiles}
\begin{document}

\subsection{Analysis of the two Methods}\label{stackBetter}
When initially parsing the infix expression $f(x)$, the binary tree version will have time complexity $O(n)$ where $n$ is the number of operators ($n$ will be the number of operators, and $m$ will be the number of operands in $f(x)$) in $f(x)$. The stack based version will convert the infix expression into a binary tree, and then perform depth-first traversal to convert the tree into post-fix. Depth-first traversal has a time complexity of $O(m+n)$ (every traversal visits every node, and there are $n+m$ nodes). This means that to initially parse the infix expression, the binary tree version has a smaller time complexity.\\
When evaluating $f(x)$ for a specific value of $x$, both versions have a time complexity of $O(m+n)$. This is because the binary tree version visits every node, $m+n$ (what we do is essentially post-order depth-first traversal but applying an operation each time), and the stack based version has $m+n$ items in the list that it goes through. Here we are assuming that each operation between operands takes equal time. This is a reasonable assumption to make because we are comparing two algorithms and they have the same input, $f(x)$, therefore we can remove the steps that it takes to complete the operations.\\
Now from a pure time complexity point of view, we can say that the binary tree version is better as the initial parsing is quicker. However when it comes to space complexity this is not true. When evaluating $f(x)$, the binary tree version uses a lot more memory because it creates subtrees every time it calls a recursive function. On the other hand, with the stack based version, the maximum space that could be occupied is $(m+n) + m)$. $(m+n)$ is the size of the original list that we pull values from, and $m$ is the maximum size our stack could ever get (we never add an operator to the stack). From a memory point of view a binary tree is not very efficiently stored. Most languages don't have a binary tree construct, and implementing your own with primitive arrays (each child is an array) is inefficient. For example the binary tree,\\
\par
\Tree[.* 
		[.+ x 4 ]
		[.- x 3 ] 
]
\bigskip \\
would be represented as (actual implementation would have pointers),
\[[* [+ [x,4] ],[- [x,3] ] ]\]
If we then stored this in contiguous memory,
\[\begin{bmatrix}
0 & 1 & 2 & 3 & 4 & 5 & 6 \\
* & + & x & 4 & - & x & 3
 \end{bmatrix}\]
Now when doing post-order depth-first traversal, we jump between the memory locations,
\[\begin{bmatrix}
2 & 3 & 1 & 5 & 6 & 4 & 0 \\
x & 4 & + & x & 3 & - & *
 \end{bmatrix}\]
This is horribly inefficient because our stride is not consistent. This is unlike our stack based version, where we always have a stride of $1$, no matter what. While this is not that big of a problem for one evaluate call, when $2000$ evaluate calls are made, this makes a massive impact. We could, in theory, modify the order that we store our tree into one which is efficient when doing a depth-first traversal but this is what we did when we converted our tree into post-fix notation. Therefore using stacks to process input is best for performance even though, the initial parsing takes longer. However this does not mean that we are not going to consider binary trees. This is because we need binary trees to initially parse our infix expression, therefore we will use stacks with binary trees to parse our user input. We need to implement trees and stacks in Java before we can start parsing our infix expression.
\newpage

\end{document}