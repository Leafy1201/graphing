\documentclass[../../../../main.tex]{subfiles}
\begin{document}

\section{Graphical User Interface}
The Graphical User Interface will be achieved using one the many toolkits within \texttt{Java}. Specifically I will be using \texttt{JavaFX\cite{javafx}}. \texttt{JavaFX} is unique in the fact that the GUI can be generated using two approaches:
\begin{enumerate}
\item Within the code itself, we can define how the GUI\cite{javafxEx} is comprised. This is done by instantiating Objects of specific component Classes, such \texttt{Pane}, \texttt{Label}, \texttt{Button} etc.\ all inheriting a top class called \texttt{Node}, and using their methods to join them together. The advantage of this approach is that since these are simply classes, we can create child classes from them allowing us to create custom components that we can control.
\item The UI is defined using \texttt{.xml} files (aptly named \texttt{FXML\cite{fxml}} files). A controller class is defined alongside it. This controller class is referenced within the the \texttt{FXML} file and when the \texttt{FXML} file is loaded, the controller class is initialized along with it. This controller allows us to ``control'' the UI nodes. While this approach is very simple to implement and the design of the UI looks clean, it can be somewhat convoluted to allow the controller classes to interact with other controller classes.
\end{enumerate}
The real strength of \texttt{JavaFX} however comes from the fact that these two approaches can be intertwined as needed, making solutions unique and versatile. 

The GUI can be decomposed into smaller parts as shown by the hierarchical diagram shown below. The parts in \textbf{bold} are the parts to be implemented and designed in Prototype I and the rest will be completed in Prototype II.

\begin{figure}[H]
\begin{center}
\begin{forest}
  for tree={
    align=center,
    calign=child,calign child=2,
    parent anchor=south,
    child anchor=north,
    font=\sffamily,
    edge={thick, -{Stealth[]}},
    l sep+=10pt,
    edge path={
      \noexpand\path [draw, \forestoption{edge}] (!u.parent anchor) -- +(0,-10pt) -| (.child anchor)\forestoption{edge label};
    },
    if level=0{
      inner xsep=0pt,
      tikz={\draw [thick] (.south east) -- (.south west);}
    }{}
  }
  [Graphical User Interface
    [\textbf{Drawing Functions}, name=A
      [\textbf{Plot Pane}
        [\textbf{Layer System}
          [\textbf{Draw onto Canvas}]
          [\textbf{Clear the Canvas}]
          [\textbf{Bind Properties}]
        ]
      ]
    ]
    [Shared Layer Access, name=C]
    [Inputing Functions, name=B
      [Add Functions]
      [Remove Functions]
      [Modify Functions]
    ]
  ]
\draw[-latex] (A) to[out=east,in=west] (C);
\draw[-latex] (B) to[out=west,in=east] (C);
\end{forest}
\end{center}
\caption{Graphical User Interface Hierarchical Diagram}
\end{figure}
Many of the features above will be discussed in greater detail later, however the Shared Layer Access and Properties will be touched upon now.

As mentioned before it can be hard to make two controller classes interact, as such one solution\cite{sharedAccess} is an object that both controller classes can reference. This class will contain attributes and methods that both classes need to interact with each other.

However, how do these classes know when something changes and hence know when to update their corresponding UI elements. The answer is through a class called \texttt{Property\cite{property}}. A \texttt{Property} can bind another \texttt{Property}, one updating when the other one changes. Objects called \texttt{Listener} can also be added to properties. A \texttt{Listener} can be programmed to do a specific action when a \texttt{Property} is updated. This action could be a function, or updating a variable. Using the class \texttt{Property} allows for a responsive UI, updating whenever the user does an action.
\newpage

\end{document}