\documentclass[../../../../../../main.tex]{subfiles}
\begin{document}

\subsubsection{Coordinate Systems}
The coordinate system that we are familiar with is the Cartesian Coordinate System. Cartesian Coordinates take a fixed point as reference, usually called the origin (symbolized by $O$), and specify a point by taking the distance from the origin in $n$ directions, where $n$ is the dimension of the system. The directions are vectors, that are all perpendicular to each other, and have magnitude 1, i.e.\ unit vectors. This graphing software will be dealing with a 2-Dimensional Cartesian Coordinate System and in this system, up and down are defined as the $y$ direction, with up being positive and down being negative; whereas right and left are defined as the $x$ direction, with right being positive and left being negative. Within our graphing program we will only show a section of the \xyplane. This is because:
\begin{enumerate}
\item It is intractable to show the entire \xyplane since the plane goes on for an infinite distance in both the $x$ and $y$ directions.
\item It is not helpful to the user to see the entire plane even if they could. A graphing tool should enable the user to highlight parts of the graph that is important to them.
\end{enumerate}
However a \texttt{JavaFX Canvas} doesn't exactly use a Cartesian Coordinate System. It specifies a position on itself taking the top-left corner as the origin, right as the positive $x$ direction and down as the positive $y$ direction. It is important to realize that negative coordinates do not exist in this system. The maximum $x$ and $y$ values are the width and height, respectively, in pixels.

We must therefore find a way to convert between these two coordinate systems. Since our viewport is predefined, let us label what the maximum and minimum $x$ and $y$ values in our viewport (with respect to our original Cartesian Coordinate System). Let these values be \texttt{minX}, \texttt{maxX}, \texttt{minY} and \texttt{maxY}.\\
Now let us define how much the width and height of each pixel on the canvas is \texttt{``worth''}. This \textit{``worth''} specifies how much going across or down a pixel width or height, is in terms of our original Cartesian Coordinate System. This is best demonstrated with an example. If our canvas shows the \xyplane in the $x$ direction from 0 to 10, and our canvas is 1000 pixels wide, then our pixel worth is the amount we have traversed in the $x$ direction, divided by the number of pixels, in this case $\frac{(10-0)}{1000} = 0.01$. The algorithm for this is shown below:

\begin{algorithm}[H]
\DontPrintSemicolon
\caption{Calculate the \textit{``Worth''} of each Pixel}
\Fn{updatePixelWorth()}{
	pixelWorthX = (maxX - minX)/canvas.getWidth()\;
	pixelWorthY = (maxY - minY)/canvas.getHeight()\;
}
\end{algorithm}
Now that we have the pixel worth, we can convert between our coordinate systems. To do this we take the vector from the origin, $O=(x_{min},y_{max})$, to the point, $P=(x,y)$ in the $x$ or $y$ direction. We then calculate how many pixel \textit{``worths''} this is and this gives us the pixel on the canvas that this coordinate is at.
It is important to realize that going down in a \texttt{JavaFX Canvas} is actually positive, relative to the canvas, hence our two algorithms are slightly different. This means our vector is actually 
\[\vec{d} = 
\begin{bmatrix}
(x-x_{min}) \\
(y_{max}-y)
\end{bmatrix}\]
Hence our position within the canvas is:
\[\vec{p} =
\begin{bmatrix}
\left(\frac{x-x_{min}}{x_{worth}}\right) \\
\left(\frac{y_{max}-y}{y_{worth}}\right)
\end{bmatrix} 
\]
The algorithms below convert between $x$ and $y$ coordinates
\footnote{Here we assume the coordinate is inside the viewport. Realistically we should check if the coordinates are in the viewport, i.e.\ bigger than the width or height of the canvas, or in some cases negative. However it does not matter since we can make this check later. Even then a \texttt{JavaFX Canvas} will check if the coordinates are in the canvas viewport anyway, so it does not matter to us since this detail is abstracted away from us.}
 respectively. 

\begin{algorithm}
\DontPrintSemicolon
\caption{Convert the $x$ Coordinate}
\Fn{convertX(\KwDouble x)}{
	x = x - minX \tcc{Calculate the horizontal distance between the point and origin}
	x = x / pixelWorthX \tcc{Calculate the number of pixel ``worths'' this is}
	\KwRet x\;
}
\end{algorithm}
\begin{algorithm}
\DontPrintSemicolon
\caption{Convert the $y$ Coordinate}
\Fn{convertX(\KwDouble y)}{
	y = maxY - y \tcc{Calculate the vertical distance between the point and origin}
	y = y / pixelWorthY \tcc{Calculate the number of pixel ``worths'' this is}
	\KwRet y\;
}
\end{algorithm}
\newpage 
\end{document}