\documentclass[../../../../../../main.tex]{subfiles}
\begin{document}

\subsubsection{Drawing the Function}
From the analysis (Section ref{sec:computability}) I briefly described how I could approximate a curve by creating a finite number of line segments.

Firstly we take the minimum value of $x$, which in the viewport is $x_{min}$. If our function is defined by $f(x)$ then let the coordinate of the function at this value be $P_1 = (x_{min},f(x_{min}))$. We then take a second coordinate a small value $dx$ away, so let the coordinate be $P_2 = (x+dx,f(x+dx))$. We now draw a line between these two points. We then make $P_1 = P_2$ and make $P_2$ with $x+2\cdot dx$. We repeat this process until $x+n\cdot dx \geq x_{max}$. We obviously need to convert the $x$ and $y$ values from the Cartesian coordinates into the canvas coordinates do we can do this when drawing the line. $dx$ will also be a finite value that is very small. We will define this by taking the number of line segments we want, and working out $dx$ for that number of steps. This is simply done by finding out how much each step is \textit{''worth''} within the viewport, i.e.\ $dx = (\texttt{maxX} - \texttt{minX})/\texttt{steps}$. The algorithm for this is shown below.

\begin{algorithm}
\DontPrintSemicolon
\caption{Draw a Function in the Viewport}
\label{alg:drawFunc}
\Fn{draw()}{
	\KwDouble x1 = minX\;
	\KwDouble y1 = f.evaluate(x1)\;		
	\KwDouble dx = (maxX - minX)/steps\;
		
	\For{x2=minX + dx \KwTo maxX \KwBy dx} {
		y2 = f.evaluate(x2)\;
		drawLine(convertX(x1), convertY(y1), convertX(x2), convertY(y2))\;
		x1 = x2\;
		y1 = y2\;		
	}
}
\end{algorithm}

A precondition of this is that \texttt{f} is a continuous function and as such we need to make sure we do not draw an undefined part of the graph, but this can be sorted out in the implementation.
\subsubsection{Input Layer}
While I will not design nor implement the responsive user input in this prototype, I will implement a basic input layer class with simple attributes.  This is because it controls the other layers since if it zooms or pans around the viewport changes and this affects the properties defined in methods in the previous sections. Properties such as the minimum and maximum values of $x$ and $y$ for the canvas, and the pixel \textit{``worth''}. So we should create a basic input layer so I can bind the properties of other layers to the input layer, which means that it is easier to build on in prototype 2.\\
This class will be quite bare, it will contain the methods to calculate pixel \textit{``worth''} and update it whenever the canvas changes size. It will be updated in other scenarios when we actually start manipulating the viewport in prototype 2, but at the moment I will simply add a property called \texttt{changeViewport} that will eventually be connected to the \texttt{PlotPane} to notify it when any input is registered. The class diagram for this is included in the next section with all the rest of the layer classes.
\newpage
\end{document}