\documentclass[../../../../../main.tex]{subfiles}
\begin{document}

\subsection{Layer System}
Drawing onto the screen using JavaFX can be achieved using the Canvas\cite{canvas} object. This is a node, which allows us to draw shapes, lines, arcs, on to the screen. As discussed in the analysis, we can approximate a curve by drawing many lines however one of our requirements was to draw multiple functions. From the article cited above, there are two ways to achieve this:
\begin{enumerate}
	\item Draw every function onto one canvas and add this to the root pane.
	\item Create a canvas for every function and draw each function onto the corresponding canvasses and add all these canvasses onto the root pane. One canvas will be paired with one function and bundled ``Layer''. This only works because canvasses are by default transparent.
\end{enumerate}
The advantage of the first is that it is simple to implement. However if there are multiple types of functions, i.e.\ an explicit function and an implicit function, then each one of these functions must have a draw function associated with it. This can become slightly convuluted, because then we must check what type of function something is, before we draw it.

Although the second is harder to implement, it is by far the superior. A canvas is essentially an abstracted OpenGL view, and hence it is almost impossible\cite{openglMultithread} to draw multiple things on the same canvas at once. However it is possible to draw onto multiple canvasses at once, which boosts performance on multicore system. This also means that we can create a class for each type of function making this system more modular. It is important to realize that only the topmost canvas receives input (keyboard and mouse events) and hence we can create a layer solely to handle input.

However before we handle the canvasses themselves, we must concern ourselves with how we can draw a line at a specific place.
\end{document}