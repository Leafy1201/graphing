\documentclass[../../../main.tex]{subfiles}
\begin{document}

\chapter{Evaluation}
Prototype 1 has been quite successful. I implemented the two features, plot a explicit function in $x$ and plot multiple functions on the same plot. I have also implemented new features. I have been able to plot an explicit function in terms of $y$ and made a Normal Distribution Function. I can plot all of the functions that I originally wanted quite accurately, with the only inaccuracies due to the programming language and not my algorithms. From this I can look at my original success criteria that are applicable here and compare and check if I have achieved them:
\begin{table}[H]
\centering
\scalebox{0.8}{
\begin{tabular}{|C{0.3\textwidth}|C{0.5\textwidth}|C{0.1\textwidth}|}
\hline
Criteria                                                                                          & Justification                                                                                                                                                                    & Achieved \\ \hline
Plots Polynomials, Exponentials and Rational Functions.                                           & These are the functions that an A-Level Student would have to know how to sketch and as such my program should be able to sketch.                                                &     \cmark     \\ \hline
The accuracy of plotting function should be comparable to other graphing programs such as Desmos. & A-Level students will be using this program and they shouldn't be misled about what a function looks like.                                                                       &      \cmark    \\ \hline
The software should be able to run on most hardware since at least 7 years ago.                   & This links to the idea that the program should be responsive but also the fact that it should be accessible to anyone who wishes to use it irrelevant of the hardware they have. &      \cmark    \\ \hline
\end{tabular}
}
\end{table}
I showed this prototype to my stakeholders and they said that they \textit{``It looks impressive so far''} and \textit{``I like the simplistic and uncluttered look of it and I hope that you keep it that way''}. Jeevon asked for a new feature where you can \textit{``hide and show any function''} and Palvinder asked for a feature where you can \textit{``see the coordinates of the point your cursor is at''}. So I will add these to the list of requirements.
\\
Sadly this project has presented some problems in terms of time. It is clear that a third prototype will not happen in the time available and some of prototype 2 will not happen either. This is the new requirements for prototype 2:
\begin{itemize}
	\item Let the user input functions
	\item Zoom in/out of the graph
	\item Pan around the graph
	\item Save plots as pictures
	\item Show/Hide functions (suggestion from a stakeholder)
	\item Show coordinates in top left of pane (suggestion from a stakeholder)
\end{itemize}
From talking to my stakeholders, these are the key things that they want in a graphing program and as such these are the things that I will definitely implement. If I have time I will implement the other features.
\newpage
\end{document}