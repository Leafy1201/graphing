\documentclass[../../../main.tex]{subfiles}

\begin{document}
\chapter{Stakeholders}
My stakeholders will be comprised of teachers and students in year 10 and above. This is because graph sketching as a skill becomes very important from GCSE onwards and as such the tool will be aimed towards students in the upper years and the teachers themselves. I have interviewed several of my potential stakeholders about the software that they use and their opinions on it. 
\section{Teacher Interview}
I interviewed Miss Naguthney who is a Maths teacher who teaches pupils from year 7 to 13. She uses multiple tools to aid her teaching. I asked her a couple of questions to get an idea of what she finds important in a graph drawing software.

\begin{figure}[H]
\centering
	\begin{tabular}{|L{0.15\textwidth}|m{0.25\textwidth}|L{0.5\textwidth}|}
	\hline
	Question & Answer & Analysis \\
	\hline
	What graph drawing software have you used? & I have used Desmos, MATLAB and GeoGebra. & These three programs are very different, ranging from a simple graph drawing software, Desmos, to a professional data presentation software, MATLAB.
	 \\ \hline
	Which do you think is the best and why? & I think all 3 are good. In terms of pure graph drawing I think
	that Desmos is the best due to its simplicity. & From this response, it is important that my program is as simple as possible so that anyone can use. I think that to make program simple it must be responsive as well, and hence I should make my solution as efficient as possible so that it can be responsive on most devices. \\ \hline
	What do you think is the most important
	aspect of a graph drawing software? & The most important part of a graph drawing software are obviously the graphs themselves and the most important part of the graphs are of course their intersections with axes, turning points and any asymptotes. & I agree with all these points made and I think that my program should at least be able to identify the turning points and any intersections with the axes.\\
	\hline
	\end{tabular}
\end{figure}


\newpage
\section{Student Interview}
I find that during my sixth form studies, that graph sketching is a must have skill, and hence a graph drawing tool is important to verify that you have drawn a graph accurately. I interviewed Matthew, who is a sixth form student studying similar subjects, about graph drawing tools that he has used and his opinions about them.

\begin{savenotes}
	\begin{figure}[H]
	
	\centering
		\begin{tabular}{|L{0.15\textwidth}|m{0.5\textwidth}|L{0.25\textwidth}|}
		\hline
		Question & Answer & Analysis \\
		\hline
		What graph drawing software have you used? &  
		GeoGebra, Desmos and Kalgebra (KDE Application) & 
		These three programs are quite similar as they have many of the same features.
		 \\ \hline
		Which do you think is the best and why? & 
		I would say GeoGebra because:
		\begin{itemize}
		\item Native client so it is more responsive than a web-based app
		\item Versatile
		\item Easily Adjustable Axes
		\item Very easy to focus on a part of the graph 
		\item Multiple function support
		\end{itemize}
		However the UI looks ugly and has no dark mode to reduce strain on the eyes.
		&
		I think that my program, should have the ability to be themed, through using CSS or a text file and while I may not be able to make my program versatile, it should still be able to draw multiple functions at once and navigating the graph must be fluid.
		\\ \hline
		What do you think is the most important
		aspect of a graph drawing software?
		&
		The two most important aspects for me are ease of use and flexible input. Specifically this would be stuff like being able to input complex functions such as sums of multiple rational functions
		\footnote{A function $f(x) = \frac{P(x)}{Q(x)}$, where $P(x)$ and $Q(x)$ are functions of $x$} (Not complex as in $x\in \mathbb{C}$)
		and the ability for the software to automatically adjust the scale. An example of this is for trigonometric functions sine and cosine. They don't have high $y$ values but they always appear very small on GeoGebra because the scale is wrong. A way to automatically set a suitable scale would be nice. Finally, I feel performance is also a must - it's frustrating to navigate the graph and have the application lag a lot.
		&
		Matthew's answer can be summarized into two points: Responsiveness and Flexible Input. I think that both of these points are valid and I should aim to make sure my application satisfies both these points. \\
		\hline
		\end{tabular}
	
	\end{figure}
\end{savenotes}


\end{document}