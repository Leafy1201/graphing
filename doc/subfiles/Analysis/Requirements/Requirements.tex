\documentclass[../../../main.tex]{subfiles}
\begin{document}
\chapter{Requirements}
From my research and analysis I have created a set of requirements that I aim to fulfill. I have split these requirements into development iterations that they will be linked with. Each iteration has a theme that most of the requirements will follow.
\begin{itemize}
	\item Iteration 1 - Emphasis on Core Functionality
		\begin{itemize}
			\item Plot a explicit function in $x$
			\item Plot multiple functions
			\item Zoom in/out of the graph
			\item Pan around the graph
		\end{itemize}
	\item Iteration 2 - Emphasis on the User
		\begin{itemize}
			\item Identify roots and turning points 
			\item Plot special functions such as trigonometric functions, logarithmics, modulus functions, etc.
			\item Multiple plots that you can switch between
			\item Save plots as pictures
			\item Save workspace to resume later
			\item \LaTeX \ equation support
			\item Dark Theme
		\end{itemize}
	\item Iteration 3 - Advanced Features
		\begin{itemize}
			\item Identify intersections between functions
			\item Differentiate explicit continuous functions with respect to $x$
			\item Polar equations
			\item Implicit equations
			\item Parametric equations
		\end{itemize}
\end{itemize}
I will use \texttt{Java} to accomplish this task. \texttt{Java} is platform independent meaning that I do not have to compile for many different systems. It also means that if I want to create a mobile version I will only have to recreate the UI. Within \texttt{Java} I will use \texttt{JavaFX} for my UI. \texttt{JavaFX} allows for customisation of UI elements through \texttt{css} files, as well as an easy interface to draw objects onto through the screen, through its \texttt{Canvas} class.  I am also comfortable with the language which makes it the best language for this project.

\end{document}