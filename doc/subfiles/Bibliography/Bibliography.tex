\documentclass[../../main.tex]{subfiles}
\begin{document}

\begin{thebibliography}{9}

\bibitem{desmos}
Desmos\\
\url{https://www.desmos.com/}

\bibitem{geogebra}
GeoGebra\\
\url{https://www.geogebra.org/}

\bibitem{octave}
GNU Octave\\
\url{https://octave.org/doc/interpreter/}

\bibitem{matlab}
MATLAB - MATrix LABatory\\
\url{https://uk.mathworks.com/help/matlab/}

\bibitem{opengl}
OpenGL\\
\url{https://www.opengl.org/documentation/}

\bibitem{regex}
RegEx - Regular Expressions\\
\url{https://en.wikipedia.org/wiki/Regular_expression}

\bibitem{lua}
Lua Example\\
\url{http://www.luaj.org/luaj/3.0/README.html}

\bibitem{stack}
Stack Based Programming\\
\url{https://en.wikipedia.org/wiki/Stack-oriented_programming}

\bibitem{javafx}
JavaFX\\
\url{https://en.wikipedia.org/wiki/JavaFX#JavaFX_11}

\bibitem{javafxEx}
JavaFX example without FXML\\
\url{https://en.wikipedia.org/wiki/JavaFX#JavaFX_application_example}

\bibitem{fxml}
Creating a GUI using FXML\\
\url{https://docs.oracle.com/javafx/2/get_started/fxml_tutorial.htm}

\bibitem{sharedAccess}
Shared Access between two or more Controllers\\
\url{https://stackoverflow.com/questions/29639881/javafx-how-to-use-a-method-in-a-controller-from-another-controller}

\bibitem{property}
Property Class from JavaFX explained\\
\url{https://docs.oracle.com/javafx/2/binding/jfxpub-binding.htm}

\bibitem{javafxHierarchy}
JavaFX interface Hierarchy\\
\url{http://www.ntu.edu.sg/home/ehchua/programming/java/Javafx1_intro.html}

\bibitem{borderpane}
Border Pane Layout Explained\\
\url{http://dammsebastian.blogspot.com/2012/03/javafx-20-layot-panes-borderpane.html}

\bibitem{canvas}
Canvas Example\\
\url{https://docs.oracle.com/javafx/2/canvas/jfxpub-canvas.htm}

\bibitem{openglMultithread}
Multithreading in OpenGL\\
\url{https://stackoverflow.com/questions/11097170/multithreaded-rendering-on-opengl}

\bibitem{generics}
Generic Programming\\
\url{https://en.wikipedia.org/wiki/Generic_programming}

\bibitem{genericsJava}
Generics within Java\\
\url{https://docs.oracle.com/javase/tutorial/extra/generics/index.html}

\bibitem{duplicationJava}
Duplicating objects in Java\\
\url{https://stackoverflow.com/questions/12072727/duplicating-objects-in-java}

\bibitem{serialize}
Serialization in Java\\
\url{http://javatechniques.com/blog/faster-deep-copies-of-java-objects/}

\bibitem{byRefJava}
Passing by Reference in Java\\
\url{https://stackoverflow.com/questions/40480/is-java-pass-by-reference-or-pass-by-value}

\bibitem{javaString}
Checking String Equivalency in Java\\
\url{https://stackoverflow.com/questions/513832/how-do-i-compare-strings-in-java}

\bibitem{countInstanceStringJava}
Counting instances of a character within a String in Java\\
\url{https://stackoverflow.com/questions/275944/how-do-i-count-the-number-
of-occurrences-of-a-char-in-a-string}

\bibitem{threadCreationJava}
Costs of instantiation of threads in Java\\
\url{https://stackoverflow.com/questions/5483047/why-is-creating-a-thread-said-to-be-expensive}

\bibitem{threadStackJava}
The Thread Stack in Java\\
\url{https://stackoverflow.com/questions/36898701/how-does-java-jvm-allocate-stack-for-each-thread}

\bibitem{threadCreationRate}
Thread Creation Rate in Java\\
The article - \url{https://stackoverflow.com/questions/2117072/java-thread-creation-overhead}\\
The benchmark I used - \url{https://stackoverflow.com/a/4371508}

\bibitem{callStack}
The Call Stack\\
\url{https://en.wikipedia.org/wiki/Call_stack}\\
\url{https://www.youtube.com/watch?v=Q2sFmqvpBe0}

\bibitem{doubleJava}
The Java implementation of the Double-precision floating-point format\\
\url{https://docs.oracle.com/javase/7/docs/api/java/lang/Double.html}

\bibitem{doubleIEEE}
The IEEE standard for undefined arithmetic behavior\\
\url{https://www.doc.ic.ac.uk/~eedwards/compsys/float/nan.html}

\bibitem{arrayListJava}
Java Implementation of a list\\
\url{https://docs.oracle.com/javase/7/docs/api/java/util/ArrayList.html}

\bibitem{listenersJava}
Listeners in JavaFX\\
\url{https://code.makery.ch/blog/javafx-2-event-handlers-and-change-listeners/}
\end{thebibliography}

\end{document}