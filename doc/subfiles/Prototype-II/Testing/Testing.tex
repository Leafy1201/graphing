\documentclass[../../../main.tex]{subfiles}
\begin{document}

\chapter{Testing}
% Please add the following required packages to your document preamble:
% \usepackage{multirow}
% \usepackage{longtable}
% Note: It may be necessary to compile the document several times to get a multi-page table to line up properly
\begin{longtable}[c]{| C{0.08\textwidth} | L{0.25\textwidth} | L{0.30\textwidth} | L{0.14\textwidth} | C{0.09\textwidth} |}
\hline
\Centering Test Number & \Centering Input                                                                                                                                  & \Centering Expected Output                                                                                                                                                                            & \Centering Actual Output & Pass/Fail \\ \hline
\endfirsthead
%
\endhead
%
1           & \textit{``e$\wedge$x"}                                                                                                                 & \multirow{4}{0.30\textwidth}{The given graphs are drawn. In test 4 the graphs should be different colours.}                                                                                                                                                &       \path{Test_1.mp4}        & \cmark    \\ \cline{1-2} \cline{4-5} 
2           & \textit{``y$\wedge$2"}                                                                                                                 &                                                                                                                                                                                            &        \path{Test_2.mp4}       & \cmark    \\ \cline{1-2} \cline{4-5} 
3           & \textit{``Normal(0,1)"}                                                                                                                &                                                                                                                                                                                            &        \path{Test_3.mp4}       & \cmark    \\ \cline{1-2} \cline{4-5} 
4           & Plot the above functions but all at once                                                                                               &                                                                                                                                                                                            &     \path{Test_4.mp4}          & \cmark    \\ \hline
5           & Keep typing \textit{``x"}                                                                                                              & The function on the plot should continually change as the input changes, here the plot will show us $x^n$ as $n$ increases by one each time                                                &       \path{Test_5.mp4}        & \cmark    \\ \hline
6           & \textit{``xy"} and \textit{``mxc"}                                                                                                     & Nothing is drawn                                                                                                                                                                           &      \path{Test_6.mp4}         & \cmark    \\ \hline
7           & \textit{``x$\wedge$2+"}                                                                                                                & While the function was being typed out, the last input that was valid should be drawn, assuming that the current input is not invalid; in this case \textit{``x$\wedge$2"} should be drawn &        \path{Test_7.mp4}       & \cmark    \\ \hline
8           & Click the add new expression button multiple times                                                                                                    & A new expression box should be added below the rest of the functions and the button should still be at the bottom. After many boxes are added a scroll bar should appear to let the user navigate between the expression boxes                                                                           &       \path{Test_8.mp4}         & \cmark    \\ \hline
9           & Click remove on any function                                                                                                           & The current function should be removed from the plot and the input pane                                                                                                                    &       \path{Test_9.mp4}        & \cmark    \\ \hline
10          & Click remove when there is only one function left                                                                                      & The current function should be removed from the plot and the input  pane and replaced with a new empty box                                                                                 &       \path{Test_10.mp4}        & \cmark    \\ \hline
11          & Click the coloured box near the input text box                                                                                         & The current function is removed from the plot, it  is still on the input pane but the coloured box is now a black outline signifying that it is hidden                                     &        \path{Test_11.mp4}       & \cmark    \\ \hline
12          & Click inside the outline of a box box near the input text box                                                                          & The current function is removed from the plot, it is still on the input pane but the black outlined box is now coloured in signifying that it is shown again                               &      \path{Test_12.mp4}         & \cmark    \\ \hline
13          & Hold left click on the plot and drag                                                                                                   & The plot appears to be dragged about the mouse                                                                                                                                             &       \path{Test_13.mp4}        & \cmark    \\ \hline
14          & Input 2 functions with intersections and over the scroll wheel up and down to zoom in and out respectively                             & The plot should appear to get closer to the intersection, zooming in and out on the spot                                                                                                   &        \path{Test_14.mp4}       & \cmark    \\ \hline
15          & Plot the two graphs \textit{``x"}and \textit{``x$\wedge$2"} and go to the intersection to verify the coordinates are working as intended & The intersection should be at $(1,1)$                                                                                                                                                    &     \path{Test_15.mp4}          & \cmark    \\ \hline
16          & Right click the plot, click save as image save it in the desired location                                                            & The saved image and the plot should be the same except for the coordinates in the corner                                                                                                   &        \path{Test_16.mp4}       & \cmark    \\ \hline
17          & Right click the plot, click save as image, and then click cancel                                                                       & The plot should be unchanged and no image should be saved                                                                                                                                  &           \path{Test_17.mp4}    & \cmark    \\ \hline
18          & Left click on a corner of the application and drag the pointer outwards or inwards to resize the window                               & The plot should remain unchanged, but should be scaled to the window, the input pane should remain the same width.                                                                     &       \path{Test_18.mp4}        & \cmark    \\ \hline
\end{longtable}
\newpage
\end{document}