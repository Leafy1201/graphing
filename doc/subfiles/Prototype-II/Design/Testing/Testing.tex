\documentclass[../../../../main.tex]{subfiles}
\begin{document}

\section{Testing Strategy}
Like in prototype 1 i will be doing unit tests however unlike prototype 1 this time the overall testing will be on the validation of input and if the implemented features actually work as intended. Each test will concentrate on one specific part of a feature and be carefully looked at to see if it is behaving as intended. I will use separate videos to record each test and these will be stored on a CD. The video evidence will be linked to each test in a separate testing table after implementation in a section of its own. If the test fails it will be analyzed and the code will be modified to fix the issue. I will also include a video of a stakeholder using it, this will be referenced in the evaluation. The testing table is shown below:

\begin{longtable}[c]{| C{0.08\textwidth} | C{0.2\textwidth} | C{0.25\textwidth} | C{0.25\textwidth} | C{0.08\textwidth} |}
\hline
Test Number & Test Description                                                                                            & Input                                                                                                                            & Expected Output                                                                                                                                                                    & Input Type \\ \hline
\endfirsthead
%
\endhead
%
1           & Plot an explicit function in terms of $x$                                                                   & \textit{"e$\wedge$x"}                                                                                                                   & \multirow{4}{*}{The given graphs are drawn}                                                                                                                                        & Valid      \\ \cline{1-3} \cline{5-5} 
2           & Plot an explicit function in terms of $y$                                                                   & \textit{"x$\wedge$2"}                                                                                                                   &                                                                                                                                                                                    & Valid      \\ \cline{1-3} \cline{5-5} 
3           & Plot a normal distribution function                                                                         & \textit{"Normal(0,1)"}                                                                                                           &                                                                                                                                                                                    & Valid      \\ \cline{1-3} \cline{5-5} 
4           & Plot multiple functions of different types                                                                  & Plot the above functions but all at once                                                                                         &                                                                                                                                                                                    & Valid      \\ \hline
5           & Plot one function and replace it, as in change the value that was input, in the same box, to something else & Keep typing \textit{"x"}                                                                                                         & The function on the plot should continually change as the input changes, here the plot will show us $x^n$ as $n$ increases by one each time                                        & Valid      \\ \hline
6           & Try to plot an invalid function i.e.\ an implicit function or letters that are not $x$ or $y$                                                  & \textit{"xy"} and \textit{"mx+c"}                                                                                                                                                                                                                                       & Nothing is drawn                                                                                                                                                                   & Invalid    \\ \hline
7           & Try to plot an incomplete valid function                                                                    & \textit{"x$\wedge$2+"}                                                                                                                  & While the function was being typed out, the last input that was valid should be drawn, assuming that the current input is not invalid; in this case \textit{"x$\wedge$2"} should be drawn & Invalid    \\ \hline
8           & Add some functions                                                                                         & Click the add new expression button & A new expression box should be added below the rest of the functions and the button should still be at the bottom                                                                 & Valid      \\ \hline
9           & Remove one function                                                                                         & Click remove on any function                                                                                                     & The current function should be removed from the plot and the input pane                                                                                                            & Valid      \\ \hline
10           & Remove a function when there is only one left                                                               & Click remove when there is only one function left                                                                                & The current function should be removed from the plot and the input pane and replaced with a new empty box                                                                          & Extreme    \\ \hline
11          & Hide a shown function                                                                                       & Click the coloured box near the input text box                                                                                   & The current function is removed from the plot, it is still on the input pane but the coloured box is now a black outline signifying that it is hidden                              & Valid      \\ \hline
12          & Show a hidden function                                                                                      & Click inside the outline of a box box near the input text box                                                                    & The current function is removed from the plot, it is still on the input pane but the black outlined box is now coloured in signifying that it is shown again                       & Valid      \\ \hline
13          & Pan around the plot                                                                                         & Hold left click on the plot and drag                                                                                             & The plot appears to be dragged about the mouse                                                                                                                                     & Valid      \\ \hline
14          & Zoom in and out of a particular spot                                                                        & Input 2 functions with intersections and move the scroll wheel up and down to zoom in and out respectively                       & The plot should appear to get closer to the intersection zooming in and out on the spot                                                                                            & Valid      \\ \hline
15          & Show how the coordinates change as the cursor moves around                                                  & Plot the two graphs \textit{"x"} and \textit{"x$\wedge$2"} and go to the intersection to verify the coordinates are working as intended & The intersection should be at $(1,1)$                                                                                                                                              & Valid      \\ \hline
16          & Save the current plot as a picture                                                                          & Right click the plot, click save as image, save it in the desired location                                                       & The saved image and the plot shoud be the same except for the coordinates in the corner                                                                                            & Valid      \\ \hline
17          & Start to save the picture as a picture but cancel                                                           & Right click the plot, click save as image, and then click cancel                                                                 & The plot should be unchanged and no image should be saved                                                                                                                          & Valid      \\ \hline
18          & Resize the application window                                                                               & Left click on a corner of the application and drag the pointer outwards or inwards to resize the window                          & The plot should remain unchanged, but should be scaled to the window, the input pane should remain the same width                                                                  & Valid      \\ \hline
\end{longtable}

There are very few invalid or extreme tests, however this makes sense since a lot of these tests were done in prototype 1 as unit tests when implementing the expression class. As such there are very few tests that I need to do to verify that the input is working properly.

\newpage
\end{document}